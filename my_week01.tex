% Options for packages loaded elsewhere
\PassOptionsToPackage{unicode}{hyperref}
\PassOptionsToPackage{hyphens}{url}
%
\documentclass[
]{article}
\usepackage{amsmath,amssymb}
\usepackage{lmodern}
\usepackage{ifxetex,ifluatex}
\ifnum 0\ifxetex 1\fi\ifluatex 1\fi=0 % if pdftex
  \usepackage[T1]{fontenc}
  \usepackage[utf8]{inputenc}
  \usepackage{textcomp} % provide euro and other symbols
\else % if luatex or xetex
  \usepackage{unicode-math}
  \defaultfontfeatures{Scale=MatchLowercase}
  \defaultfontfeatures[\rmfamily]{Ligatures=TeX,Scale=1}
\fi
% Use upquote if available, for straight quotes in verbatim environments
\IfFileExists{upquote.sty}{\usepackage{upquote}}{}
\IfFileExists{microtype.sty}{% use microtype if available
  \usepackage[]{microtype}
  \UseMicrotypeSet[protrusion]{basicmath} % disable protrusion for tt fonts
}{}
\makeatletter
\@ifundefined{KOMAClassName}{% if non-KOMA class
  \IfFileExists{parskip.sty}{%
    \usepackage{parskip}
  }{% else
    \setlength{\parindent}{0pt}
    \setlength{\parskip}{6pt plus 2pt minus 1pt}}
}{% if KOMA class
  \KOMAoptions{parskip=half}}
\makeatother
\usepackage{xcolor}
\IfFileExists{xurl.sty}{\usepackage{xurl}}{} % add URL line breaks if available
\IfFileExists{bookmark.sty}{\usepackage{bookmark}}{\usepackage{hyperref}}
\hypersetup{
  pdftitle={Homework 1},
  hidelinks,
  pdfcreator={LaTeX via pandoc}}
\urlstyle{same} % disable monospaced font for URLs
\usepackage[margin=1in]{geometry}
\usepackage{color}
\usepackage{fancyvrb}
\newcommand{\VerbBar}{|}
\newcommand{\VERB}{\Verb[commandchars=\\\{\}]}
\DefineVerbatimEnvironment{Highlighting}{Verbatim}{commandchars=\\\{\}}
% Add ',fontsize=\small' for more characters per line
\usepackage{framed}
\definecolor{shadecolor}{RGB}{248,248,248}
\newenvironment{Shaded}{\begin{snugshade}}{\end{snugshade}}
\newcommand{\AlertTok}[1]{\textcolor[rgb]{0.94,0.16,0.16}{#1}}
\newcommand{\AnnotationTok}[1]{\textcolor[rgb]{0.56,0.35,0.01}{\textbf{\textit{#1}}}}
\newcommand{\AttributeTok}[1]{\textcolor[rgb]{0.77,0.63,0.00}{#1}}
\newcommand{\BaseNTok}[1]{\textcolor[rgb]{0.00,0.00,0.81}{#1}}
\newcommand{\BuiltInTok}[1]{#1}
\newcommand{\CharTok}[1]{\textcolor[rgb]{0.31,0.60,0.02}{#1}}
\newcommand{\CommentTok}[1]{\textcolor[rgb]{0.56,0.35,0.01}{\textit{#1}}}
\newcommand{\CommentVarTok}[1]{\textcolor[rgb]{0.56,0.35,0.01}{\textbf{\textit{#1}}}}
\newcommand{\ConstantTok}[1]{\textcolor[rgb]{0.00,0.00,0.00}{#1}}
\newcommand{\ControlFlowTok}[1]{\textcolor[rgb]{0.13,0.29,0.53}{\textbf{#1}}}
\newcommand{\DataTypeTok}[1]{\textcolor[rgb]{0.13,0.29,0.53}{#1}}
\newcommand{\DecValTok}[1]{\textcolor[rgb]{0.00,0.00,0.81}{#1}}
\newcommand{\DocumentationTok}[1]{\textcolor[rgb]{0.56,0.35,0.01}{\textbf{\textit{#1}}}}
\newcommand{\ErrorTok}[1]{\textcolor[rgb]{0.64,0.00,0.00}{\textbf{#1}}}
\newcommand{\ExtensionTok}[1]{#1}
\newcommand{\FloatTok}[1]{\textcolor[rgb]{0.00,0.00,0.81}{#1}}
\newcommand{\FunctionTok}[1]{\textcolor[rgb]{0.00,0.00,0.00}{#1}}
\newcommand{\ImportTok}[1]{#1}
\newcommand{\InformationTok}[1]{\textcolor[rgb]{0.56,0.35,0.01}{\textbf{\textit{#1}}}}
\newcommand{\KeywordTok}[1]{\textcolor[rgb]{0.13,0.29,0.53}{\textbf{#1}}}
\newcommand{\NormalTok}[1]{#1}
\newcommand{\OperatorTok}[1]{\textcolor[rgb]{0.81,0.36,0.00}{\textbf{#1}}}
\newcommand{\OtherTok}[1]{\textcolor[rgb]{0.56,0.35,0.01}{#1}}
\newcommand{\PreprocessorTok}[1]{\textcolor[rgb]{0.56,0.35,0.01}{\textit{#1}}}
\newcommand{\RegionMarkerTok}[1]{#1}
\newcommand{\SpecialCharTok}[1]{\textcolor[rgb]{0.00,0.00,0.00}{#1}}
\newcommand{\SpecialStringTok}[1]{\textcolor[rgb]{0.31,0.60,0.02}{#1}}
\newcommand{\StringTok}[1]{\textcolor[rgb]{0.31,0.60,0.02}{#1}}
\newcommand{\VariableTok}[1]{\textcolor[rgb]{0.00,0.00,0.00}{#1}}
\newcommand{\VerbatimStringTok}[1]{\textcolor[rgb]{0.31,0.60,0.02}{#1}}
\newcommand{\WarningTok}[1]{\textcolor[rgb]{0.56,0.35,0.01}{\textbf{\textit{#1}}}}
\usepackage{longtable,booktabs,array}
\usepackage{calc} % for calculating minipage widths
% Correct order of tables after \paragraph or \subparagraph
\usepackage{etoolbox}
\makeatletter
\patchcmd\longtable{\par}{\if@noskipsec\mbox{}\fi\par}{}{}
\makeatother
% Allow footnotes in longtable head/foot
\IfFileExists{footnotehyper.sty}{\usepackage{footnotehyper}}{\usepackage{footnote}}
\makesavenoteenv{longtable}
\usepackage{graphicx}
\makeatletter
\def\maxwidth{\ifdim\Gin@nat@width>\linewidth\linewidth\else\Gin@nat@width\fi}
\def\maxheight{\ifdim\Gin@nat@height>\textheight\textheight\else\Gin@nat@height\fi}
\makeatother
% Scale images if necessary, so that they will not overflow the page
% margins by default, and it is still possible to overwrite the defaults
% using explicit options in \includegraphics[width, height, ...]{}
\setkeys{Gin}{width=\maxwidth,height=\maxheight,keepaspectratio}
% Set default figure placement to htbp
\makeatletter
\def\fps@figure{htbp}
\makeatother
\setlength{\emergencystretch}{3em} % prevent overfull lines
\providecommand{\tightlist}{%
  \setlength{\itemsep}{0pt}\setlength{\parskip}{0pt}}
\setcounter{secnumdepth}{-\maxdimen} % remove section numbering
\ifluatex
  \usepackage{selnolig}  % disable illegal ligatures
\fi

\title{Homework 1}
\author{}
\date{\vspace{-2.5em}}

\begin{document}
\maketitle

Used libraries are:

\begin{itemize}
  \item rethinking
  \item ggplot2
\end{itemize}

\hypertarget{question-1}{%
\subsection{Question 1}\label{question-1}}

Suppose the globe tossing data had turned out to be 8 water in 15
tosses. Construct the posterior distribution, using grid approximation.
Use the same flat prior as before.

\begin{Shaded}
\begin{Highlighting}[]
\NormalTok{p\_grid }\OtherTok{\textless{}{-}} \FunctionTok{seq}\NormalTok{(}\DecValTok{0}\NormalTok{, }\FloatTok{0.99}\NormalTok{, }\DecValTok{1}\SpecialCharTok{/}\DecValTok{100}\NormalTok{)}
\NormalTok{prior\_of\_p }\OtherTok{\textless{}{-}} \FunctionTok{rep}\NormalTok{(}\DecValTok{1}\SpecialCharTok{/}\DecValTok{100}\NormalTok{, }\DecValTok{100}\NormalTok{)}

\NormalTok{likelihood }\OtherTok{\textless{}{-}} \FunctionTok{dbinom}\NormalTok{(}\DecValTok{8}\NormalTok{, }\DecValTok{15}\NormalTok{, p\_grid) }\SpecialCharTok{*}\NormalTok{ prior\_of\_p}
\NormalTok{posterior }\OtherTok{\textless{}{-}}\NormalTok{ likelihood }\SpecialCharTok{/} \FunctionTok{sum}\NormalTok{(likelihood)}

\NormalTok{df }\OtherTok{\textless{}{-}} \FunctionTok{data.frame}\NormalTok{(}\AttributeTok{p=}\NormalTok{p\_grid, }\AttributeTok{posterior=}\NormalTok{posterior)}

\FunctionTok{ggplot}\NormalTok{(df, }\FunctionTok{aes}\NormalTok{(}\AttributeTok{x=}\NormalTok{p, }\AttributeTok{y=}\NormalTok{posterior)) }\SpecialCharTok{+} \FunctionTok{geom\_line}\NormalTok{() }
\end{Highlighting}
\end{Shaded}

\includegraphics{my_week01_files/figure-latex/unnamed-chunk-1-1.pdf}

\begin{Shaded}
\begin{Highlighting}[]
\NormalTok{samples }\OtherTok{\textless{}{-}} \FunctionTok{sample}\NormalTok{(p\_grid, }\AttributeTok{prob=}\NormalTok{posterior, }\AttributeTok{size=}\FloatTok{1e4}\NormalTok{, }\AttributeTok{replace=}\ConstantTok{TRUE}\NormalTok{) }

\NormalTok{pi }\OtherTok{\textless{}{-}} \FunctionTok{PI}\NormalTok{(samples, }\FloatTok{0.98}\NormalTok{)}
\NormalTok{posterior\_mean }\OtherTok{\textless{}{-}} \FunctionTok{mean}\NormalTok{(samples)}
\end{Highlighting}
\end{Shaded}

Posterior mean is around 0.528746 and 99 interval is between 0.26 and
0.78

\hypertarget{question-2}{%
\subsection{Question 2}\label{question-2}}

Start over in 1, but now use a prior that is zero below p = 0.5 and a
constant above p = 0.5. This corresponds to prior information that a
majority of the Earth's surface is water. What difference does the
better prior make? If it helps, compare posterior distributions (using
both priors) to the true value p = 0.7.

\begin{Shaded}
\begin{Highlighting}[]
\NormalTok{p\_grid }\OtherTok{\textless{}{-}} \FunctionTok{seq}\NormalTok{(}\DecValTok{0}\NormalTok{, }\DecValTok{1}\NormalTok{, }\DecValTok{1}\SpecialCharTok{/}\DecValTok{100}\NormalTok{)}
\NormalTok{prior\_of\_p\_blow\_05 }\OtherTok{\textless{}{-}} \FunctionTok{rep}\NormalTok{(}\DecValTok{0}\NormalTok{, }\DecValTok{51}\NormalTok{)}
\NormalTok{prior\_of\_p\_above\_05 }\OtherTok{\textless{}{-}} \FunctionTok{rep}\NormalTok{(}\DecValTok{1}\SpecialCharTok{/}\DecValTok{50}\NormalTok{, }\DecValTok{50}\NormalTok{)}

\NormalTok{prior\_of\_p }\OtherTok{\textless{}{-}} \FunctionTok{c}\NormalTok{(prior\_of\_p\_blow\_05, prior\_of\_p\_above\_05)}
\FunctionTok{names}\NormalTok{(prior\_of\_p) }\OtherTok{\textless{}{-}}\NormalTok{ p\_grid}


\NormalTok{likelihood }\OtherTok{\textless{}{-}} \FunctionTok{dbinom}\NormalTok{(}\DecValTok{8}\NormalTok{, }\DecValTok{16}\NormalTok{, p\_grid) }\SpecialCharTok{*}\NormalTok{ prior\_of\_p}
\NormalTok{posterior }\OtherTok{\textless{}{-}}\NormalTok{ likelihood }\SpecialCharTok{/} \FunctionTok{sum}\NormalTok{(likelihood)}

\NormalTok{df }\OtherTok{\textless{}{-}} \FunctionTok{data.frame}\NormalTok{(}\AttributeTok{p=}\NormalTok{p\_grid, }\AttributeTok{posterior=}\NormalTok{posterior)}

\FunctionTok{ggplot}\NormalTok{(df, }\FunctionTok{aes}\NormalTok{(}\AttributeTok{x=}\NormalTok{p, }\AttributeTok{y=}\NormalTok{posterior)) }\SpecialCharTok{+} \FunctionTok{geom\_line}\NormalTok{() }
\end{Highlighting}
\end{Shaded}

\includegraphics{my_week01_files/figure-latex/unnamed-chunk-3-1.pdf}

\begin{Shaded}
\begin{Highlighting}[]
\NormalTok{samples\_2 }\OtherTok{\textless{}{-}} \FunctionTok{sample}\NormalTok{(p\_grid, }\AttributeTok{prob=}\NormalTok{posterior, }\AttributeTok{size=}\FloatTok{1e4}\NormalTok{, }\AttributeTok{replace=}\ConstantTok{TRUE}\NormalTok{) }

\NormalTok{pi\_2 }\OtherTok{\textless{}{-}} \FunctionTok{PI}\NormalTok{(samples\_2, }\FloatTok{0.98}\NormalTok{)}
\NormalTok{posterior\_mean\_2 }\OtherTok{\textless{}{-}} \FunctionTok{mean}\NormalTok{(samples\_2)}
\end{Highlighting}
\end{Shaded}

Posterior mean is around 0.59511 and 99 interval is between 0.51 and
0.78

\begin{Shaded}
\begin{Highlighting}[]
\FunctionTok{dens}\NormalTok{( samples , }\AttributeTok{xlab=}\StringTok{"p"}\NormalTok{ , }\AttributeTok{xlim=}\FunctionTok{c}\NormalTok{(}\DecValTok{0}\NormalTok{,}\DecValTok{1}\NormalTok{) , }\AttributeTok{ylim=}\FunctionTok{c}\NormalTok{(}\DecValTok{0}\NormalTok{,}\DecValTok{8}\NormalTok{) , }\AttributeTok{show.HPDI=}\FloatTok{0.5}\NormalTok{)}
\FunctionTok{dens}\NormalTok{( samples\_2 , }\AttributeTok{add=}\ConstantTok{TRUE}\NormalTok{ , }\AttributeTok{lty=}\DecValTok{2}\NormalTok{ , }\AttributeTok{show.HPDI=}\FloatTok{0.5}\NormalTok{)}
\end{Highlighting}
\end{Shaded}

\includegraphics{my_week01_files/figure-latex/unnamed-chunk-5-1.pdf}

\hypertarget{question-3}{%
\subsection{Question 3}\label{question-3}}

This problem is more open-ended than the others. Feel free to
collaborate on the solution. Suppose you want to estimate the Earth's
proportion of water very precisely. Specifically, you want the 99\%
percentile interval of the posterior distribution of p to be only 0.05
wide. This means the distance between the upper and lower bound of the
interval should be 0.05. How many times will you have to toss the globe
to do this? I won't require a precise answer. I'm honestly more
interested in your approach.

\begin{Shaded}
\begin{Highlighting}[]
\NormalTok{f }\OtherTok{\textless{}{-}} \ControlFlowTok{function}\NormalTok{(trial)\{}
  
\NormalTok{  p\_grid }\OtherTok{\textless{}{-}} \FunctionTok{seq}\NormalTok{(}\DecValTok{0}\NormalTok{, }\DecValTok{1}\NormalTok{, }\DecValTok{1}\SpecialCharTok{/}\DecValTok{100}\NormalTok{)}
\NormalTok{  prior\_of\_p }\OtherTok{\textless{}{-}} \FunctionTok{rep}\NormalTok{(}\DecValTok{1}\SpecialCharTok{/}\DecValTok{101}\NormalTok{, }\DecValTok{101}\NormalTok{)}
  
  \FunctionTok{names}\NormalTok{(prior\_of\_p) }\OtherTok{\textless{}{-}}\NormalTok{ p\_grid}

  
\NormalTok{  likelihood }\OtherTok{\textless{}{-}} \FunctionTok{dbinom}\NormalTok{(trial}\SpecialCharTok{/}\DecValTok{2}\NormalTok{, trial, p\_grid) }\SpecialCharTok{*}\NormalTok{ prior\_of\_p}
\NormalTok{  posterior }\OtherTok{\textless{}{-}}\NormalTok{ likelihood }\SpecialCharTok{/} \FunctionTok{sum}\NormalTok{(likelihood)}
  
\NormalTok{  samples }\OtherTok{\textless{}{-}} \FunctionTok{sample}\NormalTok{(p\_grid, }\AttributeTok{prob=}\NormalTok{posterior, }\AttributeTok{size=}\FloatTok{1e4}\NormalTok{, }\AttributeTok{replace=}\ConstantTok{TRUE}\NormalTok{) }
  
\NormalTok{  interval }\OtherTok{\textless{}{-}} \FunctionTok{PI}\NormalTok{(samples, }\FloatTok{0.98}\NormalTok{)}
  
  \FunctionTok{return}\NormalTok{(}\FunctionTok{as.numeric}\NormalTok{(interval[}\StringTok{\textquotesingle{}99\%\textquotesingle{}}\NormalTok{] }\SpecialCharTok{{-}}\NormalTok{ interval[}\StringTok{\textquotesingle{}1\%\textquotesingle{}}\NormalTok{]))}
\NormalTok{\}}

\NormalTok{Nlist }\OtherTok{\textless{}{-}} \FunctionTok{c}\NormalTok{( }\DecValTok{20}\NormalTok{ , }\DecValTok{50}\NormalTok{ , }\DecValTok{100}\NormalTok{ , }\DecValTok{200}\NormalTok{ , }\DecValTok{500}\NormalTok{ , }\DecValTok{1000}\NormalTok{ , }\DecValTok{2000}\NormalTok{ )}
\NormalTok{Nlist }\OtherTok{\textless{}{-}} \FunctionTok{rep}\NormalTok{( Nlist , }\AttributeTok{each=}\DecValTok{100}\NormalTok{ )}

\NormalTok{width }\OtherTok{\textless{}{-}} \FunctionTok{sapply}\NormalTok{( Nlist , f )}

\FunctionTok{plot}\NormalTok{( Nlist , width )}
\FunctionTok{abline}\NormalTok{( }\AttributeTok{h=}\FloatTok{0.05}\NormalTok{ , }\AttributeTok{col=}\StringTok{"red"}\NormalTok{ )}
\end{Highlighting}
\end{Shaded}

\includegraphics{my_week01_files/figure-latex/unnamed-chunk-6-1.pdf}

\begin{Shaded}
\begin{Highlighting}[]
\CommentTok{\#ggplot(df, aes(x=trial, y=percentile\_interval)) + geom\_line() }
\end{Highlighting}
\end{Shaded}

\begin{Shaded}
\begin{Highlighting}[]
\NormalTok{df }\OtherTok{\textless{}{-}} \FunctionTok{data.frame}\NormalTok{(}\AttributeTok{width=}\FunctionTok{tapply}\NormalTok{(width, Nlist, }\AttributeTok{FUN=}\NormalTok{mean))}

\NormalTok{knitr}\SpecialCharTok{::}\FunctionTok{kable}\NormalTok{(df, }\AttributeTok{floating.environment=}\StringTok{"sidewaystable"}\NormalTok{)}
\end{Highlighting}
\end{Shaded}

\begin{longtable}[]{@{}lr@{}}
\toprule
& width \\
\midrule
\endhead
20 & 0.470504 \\
50 & 0.317603 \\
100 & 0.224706 \\
200 & 0.160000 \\
500 & 0.100000 \\
1000 & 0.079300 \\
2000 & 0.051112 \\
\bottomrule
\end{longtable}

\end{document}
